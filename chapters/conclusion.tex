\chapter{Closing Remarks} \label{ch:closing_remarks}
\minitoc% Creating an actual minitoc

\par Here we arrive to the end of the journey. I would like to take the space here to summarize what were the objective of my thesis and the motivation behind those objectives, what was achieved during this thesis.

% \begin{itemize}
%     \item Summarize the vision and the objectives of the PhD
%     \item
% \end{itemize}
\section{At the beginning...}

\par We started with a hypothesis that, any task human do consists of two parts: a content/identity (the core of the task) and the style (the manner the task is performed). My thesis represents our interest in studying styles, in the framework of machine learning. The reason for using machine learning is that data are becoming more complex, and applying traditional tools on it directly is no longer effective. Machine learning provide a tool for scientific research in order to explore large amounts of data; as an inquiry to see if particular information exists (as in classification), or in summarizing and compressing the data into suitable manifold, that we can perform analysis on.

\par Dealing with styles is a problem that emerges in many areas, most relevant to us, in case of human-robot interaction. Applying machine learning algorithms in a naive way directly on the data in order to learn models of human behavior leads to an averaging phenomena, where the specific style of the humans are averaged and removed. Thus, we need to find a way to extract those styles, and enable the machine learning algorithm to take into account while building models of interaction. Taking the style of the human into account during the interaction is important in order for the human to have a better experience, thus allowing some level of trust and confidence to emerge during the interaction.

\par The problem of styles is ill-defined, and poses a lot of challenges in the way we can approach it. It is not clear how to think about the problem, what framework to use, what suitable metrics to evaluate the styles, and to what benchmarks we should compare different methods. Besides, at the moment, we do not have suitable data in the area of human-robot interaction in order to performs such study. Building this data set from scratch would have been very expensive.

\par In order to have a more controlled environment, where we can study styles, we focused our attention on the problem on another problem, that we believe has relevant characteristics to our original objective: online handwriting and sketch drawing. There are several advantages of working on such domain, like the availability of large quantities of annotated data, and the task content/identity is well defined. This allowed us to focus on the problem of developing proper methods to study styles.

\par We were curious about the idea of styles in different tasks: how can we study them? is it possible to extract them? and if so, are they transferable between different tasks? Transferability of styles between task is an immensely useful idea; it can save us a lot of work in terms of collecting and annotating new data. Plus, it can allow us to better understand the common styles and aspects between different tasks.

\section{What did we do?}

\section{section name}
