\chapter{Introduction}
\minitoc% Creating an actual minitoc

\section{What is a style?}\label{sec:style}
Talk about style in images, speech, handwriting, driving,...etc. Also, distinguish between static and temporal styles.
\setlist{nolistsep}\begin{itemize}[noitemsep]
    \item What is a task? what is a style? how both combine to give us an experience?
    \item Possible scenarios: images, speech, handwriting, autonomous driving
    \item Style perception depends on 'how you look at it' (the light - angle of view -, body - task - and shadow - the resulting style - example) (cooking rice: no spices or with curcuma. perception: color? taste?) (handwriting: online styles? offline style?). Then end up with the conclusion that styles are ill-defined.
\end{itemize}

\section{Why studying styles?}
\setlist{nolistsep}\begin{itemize}[noitemsep]
    \item Shortcoming of applying machine learning methods -- average over previous scenarios, doesn't consider styles in advance --.
    \item HRI example: the need for personalization to enhance the interaction experience.
    \item Speech example: different utterances change the perceived message
\end{itemize}

\section{Why Handwriting?}
\setlist{nolistsep}\begin{itemize}[noitemsep]
    \item Availability of dataset
    \item Several style aspects are accessible to investigate
\end{itemize}

\section{An overview of the PhD}
\par The main contribution of the PhD is a thinking framework to study and observe styles, in the temporal case, using neural networks, while the task is defined beforehand. Through implementation, benchmarks and evaluation, and analysis of the networks performance (and sometimes its behaviour), we seek to provide support to ground this framework, and our hypotheses about it.

\par The most general elements/components has been used in our study, in order not to couple the effect of more advanced/complicated items with our conclusions. This leaves a big room for further improvements and exploration as well.

\par Using machine learning as tool enables us to process large amount of data.

\section{Disclaimer}
\par
