\chapter{Introduction}
\minitoc% Creating an actual minitoc

\par In one sentence, Our thesis focus on the extraction, characterization and \textit{transfer} of \textit{styles} or \textit{personas}, isolated from a task, using deep neural networks.

\section{What is a style?}\label{sec:style}
  \par Style is generically defined in Merriam-Webster dictionary (figure \ref{fig:style_def_webster}), or as \textit{the manner of doing things}~\citep{gallaher1992individual}. To get more sense of what style is, it is better to give some examples first, to get an idea about what we are dealing with.

  \begin{figure}[!htbp]
    \centering
    \boxed{\includegraphics[scale=0.5]{./images/introduction/style_def_webster.png}}
    \caption{Definition of style in Merriam-Webster dictionary}
    \label{fig:style_def_webster}
  \end{figure}

  \begin{itemize}
    \item When we say the word "seriously?". Depending on the manner we say it, it will carry different meaning (sarcastic or surprise for example). One word, two different manners to say it.
    \item Handwriting: You can write down the same the letter (the task), but with multiple typefaces (the style) (figure \ref{fig:different_fonts}).
    \item A movie setting: the script is provided to the actor. There are, however, many ways for the actor to perform what is written in the script, in order to convey different messages/experiences to the audience.
    \item Clothing and fashion: different groups of people have different general style lines -- depending on the region, ethnicity \dots etc --. Within each group, we can see people having diverse styles within this general defining style.
  \end{itemize}

  \begin{figure}[!htbp]
    \begin{center}
      \boxed{\includegraphics[scale=0.2]{./images/introduction/different_fonts.jpg}}
    \end{center}
    \caption{Multiple styles (different typefaces) for the same letters.}
    \label{fig:different_fonts}
  \end{figure}

  In these two examples, we see a basic structure:
  \begin{itemize}
    \item A fixed part: the word to be said, or the letter identity. We will call this the \textit{content}.
    \item A variable part: the manner we say the word, or the manner we write the letter. We will call this the \textit{style}.
    \item Together, a style and a content forms a \textit{task}.
  \end{itemize}

  \par The mention of styles is quite a lot in the literature in multiple domains, for example:
  \begin{itemize}
    \item \textbf{Speech synthesis}:~\citep{tachibana2004hmm} defines speaking style in a high-level manner, in terms of emotions expressed by the speaker, like 'joy', 'sad', or the interpolation between them, when reading a text.~\citep{wang2018style} looks at the speech style in a more detailed manner, considering different aspects of speech prosody, like the paralinguistic information, intonation and stress.
    \item \textbf{Car driving}: there are multiple ways to categorize the different driving styles. It can be based on the safety aspect~\citep{johnson2011driving}, aggressiveness of the maneuver~\citep{dorr2014online,xu2015establishing}, the impact on fuel consumption~\citep{manzoni2010driving,neubauer2013accounting}. Many other identification basis for driving styles are summarized nicely in~\citep{martinez2017driving}.
    \item \textbf{Handwriting}: handwriting can be offline (the final image of the letter) or online (recording the movement of the pen/drawing tool). Depending on which one considered, the style profile can change. Figure \ref{fig:different_fonts} is an example of different offline styles (the typefaces). But when we consider the online aspect of the drawing, we can see different aspects, like in figure \ref{fig:online_handwriting_styles}, where we see that the same drawing can be in clockwise or counterclockwise direction (we will expand more on this point in chapter \ref{ch:framework_sec:styleperletter}).
  \end{itemize}

  \begin{figure}[!htbp]
    \centering
    \begin{subfigure}[tb]{0.45\textwidth}
        \boxed{\includegraphics[width=\textwidth]{images/introduction/DIFFERENCE_IN_STYLE_PERCEPTION_1.jpg}}
        \caption{Offline drawing}
    \end{subfigure}

    \begin{subfigure}[tb]{0.45\textwidth}
        \boxed{\includegraphics[width=\textwidth]{images/introduction/DIFFERENCE_IN_STYLE_PERCEPTION_2.jpg}}
        \caption{Online drawing}
    \end{subfigure}
    \caption{Example for style in case of online handwriting. Although both examples looks the same when we look at it in the offline mode (the final drawing), they are quite different when we consider the online aspect (the dynamics of the pen when drawing them). Although not illustrated here, it is important to note that some aspects of the online drawing dynamics can be deduced from the offline drawing -- in other words, the dynamics can affect the end result --~\citep{diaz2017recovering}. In this case, the starting point and the direction of drawing (clockwise or counterclockwise) are different. The solid line indicate a stroke, and the dotted line indicate an air stroke (the transition of the pen in the air between two strokes). The green dot is the starting point.}
    \label{fig:online_handwriting_styles}
  \end{figure}

  \par One thing we would like to highlight here: that styles are not thing we all agree on. It can be hierarchical as well. In the clothing example earlier, people are affected by the style group they belong to, but within this cluster, people have diverse individual styles. The same happens in handwriting: education and culture affects the style cluster people belong to, but we can observe a wide range of individual styles in each cluster.

  \par Another thing to highlight is that there is no one definition for styles. It depends on the aspect of interest that we want to observe and study -- in car driving, safety and fuel consumption are two areas of interest, leading to different type of styles --. This leads to an important characteristic of \textit{styles}, that it is an ill-defined concept. We know that \textit{styles} are rich in information and important in communication between humans. They are needed in order to convey meaning. As noted in~\citep{taylor2009text} -- in the context of speech synthesis --, a proper rendering of styles affects the overall perception. However, we can not completely remove the ambiguity in this definition.

% Talk about style in images, speech, handwriting, driving,...etc. Also, distinguish between static and temporal styles.
% \setlist{nolistsep}\begin{itemize}[noitemsep]
%     \item What is a task? what is a style? how both combine to give us an experience?
%     \item Possible scenarios: images, speech, handwriting, autonomous driving
%     \item Style perception depends on 'how you look at it' (the light - angle of view -, body - task - and shadow - the resulting style - example) (cooking rice: no spices or with curcuma. perception: color? taste?) (handwriting: online styles? offline style?). Then end up with the conclusion that styles are ill-defined.
% \end{itemize}


% \section{Why studying styles?}
% \setlist{nolistsep}\begin{itemize}[noitemsep]
%     \item Shortcoming of applying machine learning methods -- average over previous scenarios, doesn't consider styles in advance --.
%     \item HRI example: the need for personalization to enhance the interaction experience.
%     \item Speech example: different utterances change the perceived message
% \end{itemize}



\section{What is the objective of this project?}
\par Our long-term objective is to enable our humanoid iCub robot Nina \ref{fig:nina_robot}, to exhibit personalized behavior suitable for the person interacting with it. This will enhance the user experience, and will allow for a more natural interaction with the robot. It is shown that a robot which exhibit a personalized behavior is more likable acceptable by humans, and perceived as an intelligent entity~\citep{churamani2017impact}. Humans have different preferences when interacting with each other, or interacting with the robot, and taking them into account does improve the quality of interaction, and the potential of success for the task~\citep{kashi2018smooth}.

\par At the moment, we successfully used machine learning approaches in order to build models of human-robot interaction~\citep{mihoub2016graphical,bailly:hal-01939223,nguyen:hal-01609535}. However, when using these models to generate behaviors, this behavior usually represents an average over the learned behaviors (which is expected). The goal is to learn models of styles, and use it to bias the models of interaction that we have, in order to generate more personalized behaviors. We want the robot to adapt to the human partner on different levels, and not just act in a reactive manner to the human actions. It has been shown that a suitable cognitive model for human-robot interaction takes into account long term styles -- for example, the person age, gender and if he/she is shy or not --, and short term decision making -- in talking with the human for example --~\citep{thorisson2002natural,bailly2010gaze}. The ability to identify, extract and use the human style traits will enable biasing the robot model to adapt for the human partner.

\begin{figure}[!htbp]
  \begin{center}
    \includegraphics[scale=0.3]{./images/introduction/nina_robot.jpg}
  \end{center}
  \caption{The iCub humanoid robot (Nina).}
  \label{fig:nina_robot}
\end{figure}

\section{Why Handwriting?}
  \par As mentioned earlier, the final objective of this project is to extract and transfer styles in the context of human-robot interaction. What this has to do with handwriting?
  \par The usage of deep learning in HRI is still in its infancy, mainly because of the lack of datasets. The issue is not collecting the data (there are many small open-source datasets available online), but having a unified set of objective and platforms for HRI, which is not an easy challenge. We envision that this problem will be resolved, as there is a growing interest in the community to address it.

  \par Thus come handwriting. We use it as a proxy platform to understand, build and test different approaches to use deep learning. It has many advantages to make it a good proxy, including:

  \begin{itemize}
    \item Availability of datasets: several datasets already exist, with large quantities of data.
    \item Diverse of tasks and styles: there are many tasks (letters) in handwriting, ranging from simple (like letter 'C') to complex (like letter 'E'), and the writers exhibit a diverse set of styles on the different tasks, making handwriting a good candidate to explore the problem of styles.
    \item Several style aspects are accessible to investigate visually, making it more accessible for in-depth analysis, and getting insight on how the model behaves.
    \item Several datasets provide information about the writers, like the age, handedness, gender and the origin. This data is interesting in some aspects of the style problem.
    \item We have a clear idea about the content of each task (the identify of the letter or the shape). Usually, the task is presented to us with the content and the style mixed together. How to disentangle the content from the style is an open question, and the fact the styles is an ill-defined problem makes it more ambiguous. With the assumption that we know the content, we can focus our effort on the styles problem\footnote{We will argue later that the task identity is not necessarily a good representation for the task content.}.
  \end{itemize}
  % A disadvantage of handwriting is the lack of the \textit{human interaction} aspect in handwriting.
  While the subjects are interacting with the environment (the pen, tablet \dots etc) -- which is observed and recorded --, there are no human interaction aspects in this domain of data, which is a disadvantage. However, style embeddings are likely to bias a sequence-to-sequence mappings in the same way.

\section{What is transfer learning? and why do we need it?}
\par We will discuss transfer learning in more detail in chapter \ref{ch:seat}, but for now, we want to motivate having this as one of the PhD objectives.

\par Transfer of knowledge deals with the problem of leveraging the knowledge learned from one task, to accelerate/improve the learning of a new task. This is a skill humans do naturally, for example, if you learn Mathematics, and you want to learn Physics, you can easily leverage the knowledge of Mathematics to bootstrap your learning of Physics. This, intuitive as it seems, is not straightforward for machine learning models. A change in the distribution of the input to the model leads to significant degradation in performance~\citep{shimodaira2000improving}.

\par Transfer learning is thus a field of machine learning, concerned with developing algorithms and procedures, to enable the transfer of knowledge between different tasks. Many techniques are available for transfer learning, but there is always a common assumption, that the transfer has the potential of success if the tasks are related (i.e., if there is common knowledge between the tasks), otherwise, a transfer learning can at best lead to no improvement, or even reduce the performance of the new model~\citep{weiss2016survey}.

\par Why do we need transfer learning? We do not always have the availability of a large datasets on the tasks that we want. In many cases, the acquisition and/or the annotation of large dataset can be prohibitively expensive. For example:
\begin{itemize}
  \item In robotics~\citep{konidaris2012robot,Konidaris:2012:TRL:2188385.2343689}, collecting data can be quite expensive process, due to hardware limitations from one side, and human limitation as well (in case of human-robot interaction scenarios). In addition, with techniques like reinforcement learning~\citep{sutton2018reinforcement}, where the robot learns by trial and error, the process can be prohibitively slow, with safety concerns sometimes. Also, no data augmentation techniques does exist in the literature for HRI, thus synthesizing extra data is not possible. Thus, we need to be able to transfer the knowledge from the task where we have large amount of data, to a relevant task where we do not have this advantage (figure \ref{fig:illustrate_TL}).
  \item In underwater acoustics~\citep{malfante2018automatic}, an essential task is collecting and cleaning the data about the different fish sounds. This is a tediously manual job, and any change (type of fish, time of the day or place in the ocean) degrades the quality of prediction a lot. Transfer learning can be very useful in this case, to reduce the effort needed to collect, clean and annotate the data.
\end{itemize}

\par What we want to achieve in this thesis is to transfer the styles between different tasks. We hypothesize that, when the tasks are relevant (but different tasks, with different contents), that humans share leverage styles between the different tasks. In case of handwriting for example, we can reuse the strokes that we learn in the uppercase letters in order to learn the lowercase letters and digits. We will test this hypothesis in details in chapter \ref{ch:seat}.

\begin{figure}[!htbp]
  \centering
  \boxed{\includegraphics[scale=0.8]{./images/introduction/transfer_learning_illustration.jpeg}}
  % \caption[Transfer Learning Illustration]{Illustration for a practical case of using transfer learning, where we have insufficient data on the new task, and we want to leverage the knowledge learned from another relevant task in order to learn the new task\footnote{Source of this image is \url{https://medium.com/data-science-101/transfer-learning-57ce3b98650}}.}
  \caption{Illustration for a practical case of using transfer learning, where we have insufficient data on the new task, and we want to leverage the knowledge learned from another relevant task in order to learn the new task. Source of this image is~\citep{transimage}.}
  \label{fig:illustrate_TL}
\end{figure}
% \footnotetext{}

\section{If we want to perform transfer learning, why extracting styles?}
  \par It is important not to forget our goal while being consumed by the shiny light of machine learning. It is easy to fall into the trap of "getting the extra 0.1\% accuracy", figure \ref{fig:current_ml_comic}. Extracting styles enables us to have an idea of what the model actually learned (i.e., it makes the model more interpretable). But why do we need interpretability? Many reasons:
  \begin{itemize}
    \item Debugging: Neural networks are notoriously known for being black box models. Extracting styles gives us some indication on what the system learned, and whether it learned relevant information about the task.

    \item Discovery of new things: in the era of big data, trying to reason on the data directly is no longer feasible. Instead, a good approach is to reason on a model that fits these data (i.e., the model compresses the data into fewer dimensions). We would like to use this model to reason and discover new things about the data. After all, \textbf{the goal of science} is to gain knowledge about the world, and an understanding of how it works. One thing we are interested in is to illuminate the workspace of interaction: what are the limits of possible actions, expected reactions and end results. These kind of questions will benefit a lot understanding what the model actually learned.

    \item Understanding why: in some cases (e.g., in case of unexpected events), it is human curiosity to understand the reasoning behind the different decisions. We would like to get insight on why the model led to that particular decision.

    \item Developing safety measures: in many applications (like when dealing with the robots), it is important to be sure that the robot is a 100\% safe for the human. Understanding what the model learned can give us insight on the shortcomings of the model, allowing us to fix or improve.

    \item Social acceptance: humans always tries to attribute beliefs, intentions, personality traits and desires to different objects~\citep{heider1944experimental}. An interpretable machine will reinforce these sensations in humans.

    \item Improve social interactions: when the robot can explain itself and its perception of the world, it creates a common understanding with the human. This allows the humans to build a mental model for what the robot is actually trying to do, thus, building trust between humans and the robot.
  \end{itemize}

  \par For further details, we strongly recommend the book~\citep{molnar2019} on the topic of machine learning interpretability.

  \begin{figure}[!htbp]
    \centering
    \boxed{\includegraphics[scale=0.5]{images/introduction/current_ml_tradition.png}}
    \caption{Source: \url{https://xkcd.com/}}
    \label{fig:current_ml_comic}
  \end{figure}


\section{Contributions of this PhD}
\par In this manuscript, we discuss the different contribution of this PhD, addressing different aspects of the styles:
\begin{enumerate}
  \item Propose a manner of thinking about styles: traditionally, a lot of work has been done in order to manually extract and annotate styles, and deal with styles in terms of predictability (using the extracting styles in a regression/classification problem). In this PhD, we propose to implicitly evaluate styles by observing the generative aspects of the model (letting the model generates behaviors, and trying to evaluate the distance between these behaviors and the target/ground truth ones). This is discussed in chapter \ref{ch:GBEM}.
  \item Propose and build benchmarks and evaluation metrics, and ground those metrics, in order to compare and evaluate future style extraction methods. This is discussed in chapter \ref{ch:GBEM}.
  \item Propose a generic framework to study styles, using a conditioned-autoencoder. We evaluate this framework in its basic form against the benchmarks. We further validate this framework by extracting verbose styles from it, including ones that are not known from the literature. This is discussed in chapter \ref{ch:framework}.
  \item Last, we address the problem of style transfer. We show how to use our proposed framework in order to transfer styles. We perform extensive experiments in a lot of combinations of tasks, on two different datasets. We also enhance and expand on the evaluation metrics we use in order to quantify the quality of transfer. This is discussed in chapter \ref{ch:seat}.
\end{enumerate}
% Some informal contributions are: \textbf{not sure about this section}
% \begin{enumerate}
%   \item Definition of styles??
% \end{enumerate}

% \section{An overview of the manuscript}
% \par The main contribution of the PhD is a thinking framework to study and observe styles, in the temporal case, using neural networks, while the task is defined beforehand. Through implementation, benchmarks and evaluation, and analysis of the networks performance (and sometimes its behavior), we seek to provide support to ground this framework, and our hypotheses about it.
%
% \par The most general elements/components has been used in our study, in order not to couple the effect of more advanced/complicated items with our conclusions. This leaves a big room for further improvements and exploration as well.
%
% \par Using machine learning as tool enables us to process large amount of data.

\section{Thesis outlines}
\par We start in chapter \ref{ch:dataset} by explaining the datasets used in this PhD, and explore their different characteristics, and the preprocessing performed on them. In chapter \ref{ch:GBEM}, we discuss first the different aspect of deep learning that we are using in this PhD. We then discuss the different benchmarks and evaluation metrics we propose, and how can we ground them.

\par Once this step is done, it paves the way to discuss our proposed framework to study styles in chapter \ref{ch:framework}. We start first by presenting the relevant literature, then move on to the experimental part, where we show the performance of the proposed framework relative to the benchmarks. We conclude this chapter by a section on style extraction, where we shade some light on what the model actually learned.

\par With the elements in place, we can explore the topic of style transfer, in chapter \ref{ch:seat}. As usual, we study with literature over transfer learning, followed by the proposed experiments in order address our hypotheses. We perform a wide range of experiments, to solidify our conclusions about style transfer. We also discuss another possibility to interpret what the model actually learned in section \ref{ch:seat_sec:rf}.

\par We conclude the manuscript by a general discussion over this thesis (chapter \ref{ch:discussion}), the difficulties it faced, and the possible future research directions based on the results. We believe that this thesis answered multiple questions, but it also created more questions and research interests for further investigation. We then make a short summary and conclusions about the work performed in this thesis in chapter \ref{ch:closing_remarks}.
