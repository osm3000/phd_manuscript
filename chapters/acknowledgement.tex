\chapter*{A word and deserved acknowledgement}
\par Although three years, it seems like a lot of time. Many things happened, too many memories. I started the PhD with a lot of confidence in my capability to do a great work in science, with a lot of ideals that can seems naive sometimes. I was reading a lot for Richard Feynman, Stephen Hawking, Richard Hamming\footnotes{I always appreciated his article \textit{You and Your Research}, \url{https://www.cs.virginia.edu/~robins/YouAndYourResearch.html}}, John Nash, and many others.

\par I do not really consider myself as a scientist. I always identify myself as an engineer first, manager second, and science enthusiast third. I still get very excited with simple things (seeing the robot behavior after evolving its gait and morphology), and even more excited when there is an expected problem that rise (if it is not a programming or logical error, then this is awesome! it is time to do some science!!)\footnotes{Thanks to my awesome Master-1 internship supervisor, Jean-Baptiste Mouret (INRIA Nancy).}. I love problems and puzzles, I love tackling them, analyzing them, looking at them from different angles. I do not care much about what is that called.

\par Things did not exactly go as expected though, for reasons that are not its place here to discuss. Not far from the starting line, things went from dreamy science to crises management. I do not want to complain much either, I have made my peace with the past, and it is time to move on.

\par I think there are three main aspects that I think I won from the PhD, which are completely different from my earlier expectations:
\begin{itemize}
  \item I started to value the people around me much more than before. It was a hard time for me on many levels. I am blessed and grateful for the amazing friends I was so lucky to have. They provided solid support system. Without them, I can not imagine going through this experience. It made it special.
  \item I had time to tinker with different problems. Even without authorization, I was experimenting on my own ideas from time to time. Some results were quit interesting, and I plan to pursue them further after the PhD.
  \item In my whole life, work has always taken the first priority, since my first grad in primary school, till most of the time in the PhD. This finally broke down, and I am happy it happened. I am mentally free. I am happy to discover and appreciate more the human aspect in my life, and learning how to enjoy life and appreciate the current moment, more than being enslaved to work.
\end{itemize}

\par One of the amazing things that happened during the PhD was that we orgranied a machine learning club, \textit{Machine Learning Seekers of Truth} (MLST), with Marielle Malfante, Gael Le-Godais, Fanny Roche, Branislav Gerazov (Branko), Ludovic Darmet, Julien Muzeau, Dawood El-Chanti, Pedro Rodrigues, Thomas Hueber, and others who joined later. I hope this club will be revived again by the new PhD students, since most of the core people either finished or getting busy with the third year.

names:

Gael LeGodais
Remi Cambuzat
Firas
Andrei
Sophie
Anne
Marielle
Fanny
Baharat
Alex
Matthieu
Ludovic
Julien
Thibaut
Helene
Duc-Canh
Branko
Dan
Jean-Francois
Quan
Insaf
Amgad
Kaouther
Makia
Aida
Jeanne
Dawood and Sami
Jodie
Christine
------------------
from the other life
Abdien
Hassan
Kassem
Dr.Abdelsalam
mentor graphics people
Moghany
Dola
