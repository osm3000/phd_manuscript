\chapter*{A word and deserved acknowledgement}

\par Those three years are a lot of time. Many things happened, too many memories. I started the PhD with a lot of confidence in my capability to do a great work in science, with a lot of ideals that can seem naive sometimes. I was reading a lot for Richard Feynman, Stephen Hawking, Richard Hamming\footnote{I always appreciated his article \textit{You and Your Research}, \url{https://www.cs.virginia.edu/~robins/YouAndYourResearch.html}}, John Nash, and many others.

\par I do not really consider myself as a scientist. I always identify myself as an engineer first, manager second, and science enthusiast third. I still get very excited with simple things (seeing the robot behavior after evolving its gait and morphology), and even more excited when there is an unexpected problem that rise (if it is not a programming or logical error, then this is awesome! it is time to do some science!!)\footnote{Thanks to my awesome Master-1 internship supervisor, Jean-Baptiste Mouret (INRIA Nancy), for teaching me this attitude.}. I love problems and puzzles, I love tackling them, analyzing them, looking at them from different angles. I do not care much about what is that called.

\par Things did not exactly go as expected though, for reasons that are not its place here to discuss. Not far from the starting line, things went from dreamy science to crisis management. I do not want to complain much either, I have made my peace with the past, and it is time to move on.

\par I think there are three main aspects that I think I won from the PhD, which are completely different from my earlier expectations:
\begin{itemize}
  \item I started to value the people around me much more than before. It was a hard time for me on many levels. I am blessed and grateful for the amazing friends I was so lucky to have. They provided solid support system. Without them, I can not imagine going through this experience. It made it special.
  \item I had time to tinker with different problems. Even without authorization, I was experimenting on my own ideas from time to time. Some results were quite interesting, and I plan to pursue them further after the PhD.
  \item In my whole life, work has always taken the first priority, since my first year in primary school, till most of the time in the PhD. This finally broke down, and I am happy it happened. I am mentally free. I am happy to discover and appreciate more the human aspect in my life, and learning how to enjoy life and appreciate the current moment, more than being enslaved to work.
\end{itemize}

\par I would like to my thesis director, G\'erard Bailly, for his dedication during my PhD. Our relationship has gone through many phases, some of them were quite difficult. We had our good share of agreements and disagreements. That being said though, I appreciate the fact that G\'erard always allocated time whenever I need to discuss with him\footnote{Something that rarely exists for many other PhD students.}. During the third year, I was becoming increasingly tired and exhausted, and this especially became evident when I started writing this manuscript. I was literally losing it. G\'erard's support and understanding during this period was crucial. The scientific discussions were always interesting and enriching for me, given G\'erard vast experience. For all of that and more, thank you.

\par I want to thank my family, my mother for her unlimited love, support and sacrifice, my brother for taking care of the family in my absence, and especially my little sister Zynab: although 11 years younger than me, her beautiful personality and heart always filled my life with love and warmth. For that, I am grateful.

\par One of the amazing things that happened during the PhD was that we organized a machine learning club, \textit{Machine Learning Seekers of Truth} (MLST), with Marielle Malfante, Ga\"{e}l Le-Godais, Fanny Roche, Branislav Gerazov (Branko), Ludovic Darmet, Julien Muzeau, Dawood El-Chanti, Pedro Rodrigues, Thomas Hueber, and others who joined later. I hope this club will be revived again by the new PhD students, since most of the core people either finished or getting busy with the third year.

\par I am quite grateful for the discussions with Marielle, Ga\"{e}l, Fanny, Branko, Ludovic, Thomas, Dawood and others during the PhD. These discussions on the technical level were extremely rich. I can safely say that it contributed to the basis of the PhD and the choices of directions of research.

\par Thanks a lot Firas, Andrei, Makia, Sophie, Anne, Adela, Remi, Ga\"{e}l, H\'el\`ene, Fanny, Branko, Dan, Alexandre Hennequin, Matthieu, Thibaut, Kaouther, Amgad, Baharat, Christine, Sylvain Geranton, Silvain Gerber, Alexandre Serazin, Jean-Francois, Julien, Sami, Quan, Duc-Canh, Jodie, Roxi, Insaf, Katia, Alexandra, Azadeh, Analisa, Anneke and the others whom I hope I didn't forget to mention, for being such an amazing company. The words fall short in describing my feelings about you. Seeing you from day to day gave me strength and reason to continue. I am grateful for all the good experiences, memories, love and care that I enjoyed being with you. From all my heart, thank you. I wish you all the best in your lives and your future choices. The fellowship may sadly break, as we diverge in our choices in life after this stage, but our friendship will last, and I keep faith that our paths will cross again.

\par I would like also to thank Eric Cartman\footnote{From the beautiful town of \textit{South Park}.}. Eric has been a model for me, in being pragmatist, knowing how to manage complex situations, thinking out of the box, and finding opportunities and victories in the most difficult scenarios. What I liked about him is that he never distinguished between anyone, he dealt with everyone in the same way.

\par I want also to thank planet Jupiter. Jupiter is one of those heroes which we take for granted. Jupiter's massive gravity and magnetic field has protected earth since the beginning of time. We all have a chance in living because of it, thus, thanks are in order here.

\par Last but not least, I would like to thank myself for making it through. Thanks Omar :)

\par This chapter of my life is now peacefully concluded. It is time to move on :)

\begin {flushright}
  Omar Samir Mohammed\\
  27th of January, 2020\\
  Grenoble, France
\end {flushright}
